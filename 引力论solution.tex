
%!TEX TS-program = xetex
%!TEX encoding = UTF-8 Unicode


\documentclass[12pt]{report}
\usepackage{geometry}
\geometry{letterpaper}

\usepackage{fontspec,xltxtra,xunicode}
\defaultfontfeatures{Mapping=tex-text}
\setromanfont{STSong} %设置中文字体
\XeTeXlinebreaklocale “zh”
\XeTeXlinebreakskip = 0pt plus 1pt minus 0.1pt %文章内中文自动换行

\newfontfamily{\E}{Arial}  %设定新的字体快捷命令
\newcommand{\ud}{\mathrm{d}} %设置\ud为罗马字体的d

\usepackage{amsmath}

\newsavebox{\myhbar}%由于xunicode的\hbar有bug,故重新设置\hbar
\savebox{\myhbar}{$\hbar$}
\renewcommand*{\hbar}{\mathalpha{\usebox{\myhbar}}}

\usepackage{xparse}%设置双重指标
\ExplSyntaxOn
\NewDocumentCommand{\tensorkor}{ s m O{} }
 {
  \IfBooleanTF{#1}
   { \gabriel_nonstacked:nn { #2 } { #3 } }
   { \gabriel_stacked:nn { #2 } { #3 } }
 }
\cs_new_protected:Npn \gabriel_stacked:nn #1 #2
 {
  \mathsf{#1}%
  \clist_map_inline:nn { #2 }
   {
    \vphantom{\mathsf{#1}} ##1
   }
 }
\tl_new:N \l_gabriel_sup_tl
\tl_new:N \l_gabriel_sub_tl
\cs_new_protected:Npn \gabriel_nonstacked:nn #1 #2
 {
  \tl_clear:N \l_gabriel_sup_tl
  \tl_clear:N \l_gabriel_sub_tl
  \clist_map_inline:nn { #2 }
   {
    \gabriel_decide:Nn ##1
   }
  \mathsf{#1}\sp{\l_gabriel_sup_tl}\sb{\l_gabriel_sub_tl}
 }
\cs_new_protected:Npn \gabriel_decide:Nn #1 #2
 {
  \token_if_math_superscript:NTF #1
   {
    \tl_put_right:Nn \l_gabriel_sup_tl { #2 }
   }
   {
    \tl_put_right:Nn \l_gabriel_sub_tl { #2 }
   }
 }
\ExplSyntaxOff


\title{\H 引力论习题解答}
\author{EsyDragonOne}
\date{\E\today}


\begin{document}
\maketitle
\tableofcontents

\chapter{空时物理学}
\section{第一章 几何动力学概要}
\subsection{圆柱体的曲率}
试证明圆柱体表面上各短程线(不缠绕!!)无短程线偏差,并由此证明圆柱体表面的Gauss曲率R为零.再利用公式$R=1/\rho_{1}\rho_{2}$独立地证明上述结论,式中的$\rho_{1}$和$\rho_{2}$是题中相对于圆柱所在的三维Euclid空间的主曲率半径.

{\E Solution:}

在圆柱体上考虑不缠绕的短程线即为圆柱的圆截线.而考虑任意两条短程线,将圆柱体沿母线展开得到一矩形,其中圆截线则对应与矩形的边平行的直线.故任意两条不缠绕的短程线均平行,即无短程偏差.
\begin{displaymath}
\frac{\ud^{2} \xi}{\ud s^{2}}=0
\end{displaymath}
故Gauss曲率为零.

而圆柱在三维Euclid空间中的主曲率$k_{1}=1/a,k_{2}=0$,故其高斯曲率为$R=k_{1}k_{2}=0$.既有
\begin{displaymath}
\frac{\ud^{2} \xi}{\ud s^{2}}=0
\end{displaymath}
\rightline{证毕}

\newpage
\subsection{春潮与小潮}
试用约定单位制以及几何单位制计算

(1)月亮($m_{约定}=7.35\times 10^{25}g$,\quad$r=3.84\times 10^{10}cm$)在地球上产生的Newton潮汐加速度$R^{m}_{\phantom 0n0}$(m,n=1,2,3)之大小.

(2)太阳($m_{约定}=1.989\times 10^{33}g$,\quad$r=1.496\times 10^{13}cm$)在地球上产生的Newton潮汐加速度$R^{m}_{\phantom 0n0}$(m,n=1,2,3)之大小.

(3)估计对春潮所得之结果应超过小潮多少倍.

{\E Solution:}

考虑牛顿极限下有等式:
\begin{displaymath}
\begin{pmatrix}
R^{1}_{\phantom 0010} &R^{2}_{\phantom 0010} &R^{3}_{\phantom 0010}\\
R^{1}_{\phantom 0020} &R^{2}_{\phantom 0020} &R^{3}_{\phantom 0020}\\
R^{1}_{\phantom 0030} &R^{2}_{\phantom 0030} &R^{3}_{\phantom 0030}
\end{pmatrix}
=
\begin{pmatrix}
\frac{m}{r^{3}} &0 &0\\
0 &\frac{m}{r^{3}}  &0 \\
0 &0 &-\frac{2m}{r^{3}} 
\end{pmatrix}
\end{displaymath}
在约定单位下$G$取$6.674\times 10^{-8}cm^{3}g^{-1}s^{-2}$,则有当为约定单位制时:
\begin{displaymath}
\frac{GM_{sun}}{r_{sun}^{3}}\simeq 3.96\times 10^{-14}s^{-2}
\end{displaymath}
\begin{displaymath}
\frac{GM_{moon}}{r_{moon}^{3}}\simeq 8.66\times 10^{-14}s^{-2}
\end{displaymath}
转换为几何单位制时:
\begin{displaymath}
\frac{M_{sun}}{r_{sun}^{3}}\simeq 4.41\times 10^{-35}cm^{-2}
\end{displaymath}
\begin{displaymath}
\frac{M_{moon}}{r_{moon}^{3}}\simeq 9.62\times 10^{-35}cm^{-2}
\end{displaymath}
故采用几何单位制中,\\
%$\tensorkor{R_{sun}}[^1,_0,_1,_0]=\tensorkor{R_{sun} }[^2,_0,_2,_0]=4.41\times 10^{-35}cm^{-2}$\\
%$\tensorkor{R_{sun}}[^3,_0,_3,_0]=-8.82\times 10^{-35}cm^{-2}$\\
%$\tensorkor{R_{moon}}[^1,_0,_1,_0]=\tensorkor{R_{moon} }[^2,_0,_2,_0]=9.62\times 10^{-35}cm^{-2}$\\
%$\tensorkor{R_{moon}}[^3,_0,_3,_0]=-1.92\times 10^{-34}cm^{-2}$\\

而考虑最简单模型,春潮中月亮和太阳同时对潮汐加速度有贡献,而小潮仅仅是由月亮引起的.既有春潮的潮汐加速度是小潮的约1.46倍.

\newpage
\subsection{被包裹在里面的Kepler}
一个小卫星绕质量为$m(cm)$的中心物体,在半径为$r$的轨道上,以圆频率$\omega(cm^{-1})$运动.试证明,由已知的$\omega$值,不可能单独确定$r$或$m$的值,而只能得到物体的有效"Kepler密度",即以轨道半径为半径的球体平均密度.给出以Kepler密度表示的$\omega^{2}$的公式.

{\E Solution:}

由Newton第二定律可得:
\begin{displaymath}
\frac{Mm}{r^{2}}=m\omega^{2}r
\end{displaymath}
又因为由定义
\begin{displaymath}
\rho_{\kappa}=\frac{M}{\frac{4}{3}\pi r^{3}}
\end{displaymath}
所以得到
\begin{displaymath}
\omega^{2}=\frac{4}{3}\pi \rho_{\kappa}
\end{displaymath}

\chapter{平直空时中的物理学}
\section{第二章 狭义相对论基础}
\subsection*{练习2.1}
试证明式$\quad \boldsymbol{p\cdot v=\langle p\cdot v\rangle}\quad$与de Broglie波的量子力学性质一致
\begin{displaymath}
\psi =e^{i\phi}=exp[i(\boldsymbol{k\cdot x}-\omega t)]
\end{displaymath}

{\E Solution:}

我们已知
\begin{displaymath}
\begin{split}
\boldsymbol{\langle k,v\rangle}&=\phi(P)-\phi(P_{0})\\
&=\ud \phi
\end{split}
\end{displaymath}
而两边同时乘以$\hbar$
\begin{displaymath}
\begin{split}
\boldsymbol{\langle p,v\rangle}&=\boldsymbol{\langle \hbar k,v\rangle}\\
&=\hbar \ud \phi
\end{split}
\end{displaymath}

de Broglie波的动量期待值的分量为
\begin{displaymath}
\begin{split}
p^{j}=\langle \psi|p^{j}|\psi \rangle&=\int \psi^{*} \frac{\hbar}{i} \frac{\partial \psi}{\partial x_{j}} \,\ud x\\
&=\hbar k^{j} \int \psi^{*} \psi\,\ud x\\
&=\hbar k^{j}\\
\end{split}
\end{displaymath}

将$\boldsymbol{p}$与任意单位矢量做标积得
\begin{displaymath}
\boldsymbol{p\cdot v}=p^{j}=\hbar k^{j}=\hbar \ud \phi
\end{displaymath}

故可得$\quad \boldsymbol{p\cdot v=\langle p\cdot v\rangle}\quad$


\section{电磁场}
\section{电磁学和微分形式}
\section{压强-能量张量和守恒律}
\section{加速的观测者}
\section{引力与狭义相对论的不相容性}


\end{document}









